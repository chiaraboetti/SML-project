\documentclass[a4paper,10pt, oneside]{article}  % document class
\usepackage{geometry}
\geometry{
  inner=23mm,
  outer=23mm,
  top=37.125mm,
  bottom=37.125mm,
  heightrounded,
  marginparwidth=90pt,
  marginparsep=17pt,
  headsep=24pt,
}
\usepackage{hyperref}
%pacchetto
\usepackage{import, multicol,lipsum}  % package
\usepackage[utf8]{inputenc} % accenti facili
\usepackage[T1]{fontenc}
\usepackage{subcaption}
\usepackage{pifont}
\usepackage{url}
\usepackage[english]{babel} % e inglese no?
\usepackage{graphicx, color, blindtext}
\usepackage{textcomp, makeidx, times}
\usepackage{amsthm, amsmath, amssymb, amsfonts, mathtools} % math
\usepackage[mathscr]{eucal}		
\usepackage[nottoc]{tocbibind} 
\usepackage{pgfplots, parskip}
\usepackage{afterpage, ifthen}
\usepackage{enumitem}
\pgfplotsset{compat=newest}
\usepackage{graphicx} % immagini
\graphicspath{ {image/} } % path cartella delle immagini
\usepackage{tikz} %graph
\usetikzlibrary{arrows,automata}
\usetikzlibrary{automata,arrows,positioning,calc,matrix}
\usepackage[linesnumbered,ruled,vlined]{algorithm2e}
\usepackage{booktabs}
\usepackage{colortbl}
\usepackage{siunitx}
\usepackage{tabularx, tabu}
\usepackage{relsize}
\usepackage{makecell, caption, chngcntr}
\usepackage{bbm}
\usepackage{diffcoeff}
\RequirePackage{fix-cm}


%
%____________________________________________________________________________________________________________________________
%____________________________________________________________________________________________________________________________
%____________________________________________________________________________________________________________________________
%____________________________________________________________________________________________________________________________
% INIZIO
\begin{document}

\setcounter{secnumdepth}{2}
\pagestyle{plain} % stile pagina (header, numerazioni)

\centerline {\includegraphics[width=2cm]{logo.jpg}}
\begin{center}
Università degli Studi di Torino - M.Sc.  in Stochastic and Data Science - A.Y.  2021/2022

\line(1,0){450}\\ 
\vspace{0.1cm} 
{\Large \textbf{ Final project of Statistical Machine Learning (MAT0043)}} \\ 
\vspace{0.1cm}
\hspace{0cm} 
Group: Bartoli Francesco,  Boetti Chiara (854411), Bordino Alberto (856592) [questa parte e' poi da togliere]
\line(1,0){450} \\
\end{center}

%____________________________________________________________________________________________________________________________
% INTRO

\section*{Abstract}
\textit{Riassunto delle puntate precedenti:}
\begin{itemize}
\item Oggigiorno la medicina ha a disposizione una grande varieta' di dati che possono essere sfruttati per migliorare ed accelerare la ricerca. In particolare, i genome scale screens sono utilizzati nell'identificare i geni che influiscono maggiormente la crescita di cellule cancerogene cancro. \\
\item In questo progetto, applichiamo up-to-date tecniche di SML per classificare i tipi di cancro e trovare quali sono i geni piu' impattanti. Ad esempio, utilizziamo RF, NN, Lasso-SVM e Logistic-Lasso\\
\item In particolare, vediamo come la stessa tecnica di classificazione da' risultati differenti a seconda del cancro considerato. Questo perche' l'analisi di questi dati e' abbastanza tricky: c'e' una grande (grandissima) quantita' di dati, molto correlati tra loro e (altra ragione). \\
\item Non so se vogliamo gia' mettere dei risultati conclusi qua
\end{itemize}

\begin{multicols}{2}

\section{Introduction}
Leggere l'articolo di Fra. \\
Scrivere che il cancro e' una malattia molto brutta e brevemente come funziona (aka le cellule impazziscono: svillupano mutazioni, diventano imprevedibile e causano problemi negli individui...). Il compito della medicina e' trovare delle cure (banana). E qua entra in gico DepMap: DNA arrays che raccolgono mutationi dei geni delle cellule cancerogene.

\section{The dataset}


\section{Methods}


\section{Results}


\section{Commenting the results}


\section{Future works}


\section{Contribution}


\end{multicols}

\begin{thebibliography}{11}   % BIBLIOGRAFIA
\bibitem{Baum} L. Baum, T. Petrie, \textit{Statistical Inference for Probabilistic Functions of Finite State Markov Chains}, in Ann. Math. Stat., 37, 1554-1563, 1966.
\end{thebibliography}
\end{document}
