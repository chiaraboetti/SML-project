\documentclass[a4paper,11pt, oneside]{article}  % document class
\usepackage{geometry}
\geometry{
  inner=20mm,
  outer=20mm,
  top=27mm,
  bottom=25mm %
%  heightrounded,
%  marginparwidth=50pt,
%  marginparsep=17pt,
%  headsep=20pt
}
\usepackage[english]{babel}
\usepackage{hyperref}
%pacchetto
\usepackage{import, multicol,lipsum}  % package
\setlength{\columnsep}{1cm}

\usepackage[utf8]{inputenc} % accenti facili
\usepackage[T1]{fontenc}
\usepackage{subcaption}
\usepackage{pifont}
\usepackage{url}
\hypersetup{
	colorlinks=true,
	linkcolor=blue,
	filecolor=magenta,      
	urlcolor=blue
}
\usepackage{graphicx, color, blindtext}
\usepackage{textcomp, makeidx, times}
\usepackage{amsthm, amsmath, amssymb, amsfonts, mathtools} % math
\usepackage[mathscr]{eucal}		
\usepackage[nottoc]{tocbibind} 
\usepackage{pgfplots, parskip}
\usepackage{afterpage, ifthen}
\usepackage{enumitem}
\pgfplotsset{compat=newest}
\usepackage{graphicx} % immagini
\usepackage{wrapfig}
%\graphicspath{ {image/} } % path cartella delle immagini
\usepackage{tikz} %graph
\usetikzlibrary{arrows,automata}
\usetikzlibrary{automata,arrows,positioning,calc,matrix}
\usepackage[linesnumbered,ruled,vlined]{algorithm2e}
\usepackage{booktabs}
\usepackage{colortbl}
\usepackage{siunitx}
\usepackage{tabularx, tabu}
\usepackage{relsize}
\usepackage{makecell, caption, chngcntr}
\usepackage{bbm}
\usepackage{diffcoeff}
\RequirePackage{fix-cm}


%
%____________________________________________________________________________________________________________________________
%____________________________________________________________________________________________________________________________
%____________________________________________________________________________________________________________________________
%____________________________________________________________________________________________________________________________
% INIZIO
\begin{document}

\setcounter{secnumdepth}{2}
\pagestyle{plain} % stile pagina (header, numerazioni)

\centerline {\includegraphics[width=2cm]{logo.jpg}}
\begin{center}
	Università degli Studi di Torino - M.Sc.  in Stochastic and Data Science - A.Y.  2021/2022 \\
	\Large { Final project of Statistical Machine Learning (MAT0043)}
	\line(1,0){450}\\ 
	\vspace{0.4cm} 
	{ \huge \textbf{Gene selection for cancer type classification} }
	\vspace{0.1cm}
	\line(1,0){450} \\
\end{center}

%____________________________________________________________________________________________________________________________

The purpose of our project is to work on a high-dimensional genomics data and find a relatively small number of genes to predict the cancer type of a given tumorous cells. This is known as Genes Selection for Cancer Classification and it is in line with many up-to-date problems of applied medicine. \\
Our dataset contained $1.032$ cancerous cells and their knock-out probabilities, i.e. the probabilities of stopping the growth of tumor by inhibiting one of the $\sim 17.000$ genes. Each cell line was characterized by one of 10 possible cancer labels: Eye, Gastrointestinal, Gynecologic, Musculoskeletal, Neurological, Breast, Head-Neck, Hematologic, Genitourinary and, finally, Lung. We explored in details three Features Selection algorithms: Random Forests (RF) combined with Feature Importance, Lasso-SVM and Neural Networks (NN) combined with Olden Importance.  We studied two binary classification problems (Blood-cancer vs. All, Lung-cancer vs. All) and the multiclass problem. Besides lung models, we achieved satisfying classification accuracies and we were able to select about $100$-$200$ genes from the initial $\sim 17.000$ ones. Models fitted on such genes obtained classification accuracies ranging from $67\%$ to $98\%$. \\
Therefore, it seems that classifying cancer type from an extremely small set of genes depends on the cancer type itself. These methodologies worked incredibly well on Blood cancer, as  we reached almost $100\%$ accuracy with the reduced classifier,  while failed miserably on Lung cancer.  


\section{Introduction}
Cancer is a complex disease characterized by the uncontrolled growth of abnormal cells anywhere in the body. These abnormal cells are extremely invasive and we usually identify them with the name of their original tissue (for instance, breast cancer, lung cancer,  brain cancer, etc.).  In recent years, medicine has made a great step forward in finding new and efficient therapies for different diseases and cancer is one of them.  In particular,  thanks to numerous advances in technology,  collecting huge amount of data is no longer an issue, so that one can exploit such information to define personalized treatments for patients. In this regards, the DepMap project\footnote{DepMap Portal: \url{https://depmap.org/portal/} } and,  in particular,  the Achilles project\footnote{Achilles Project: \url{https://depmap.org/portal/achilles/} } aim to use genome-wide screens to collect data regarding mutations of cancerous cells,  identify essential genes and report vulnerabilities across hundreds of human cancers.  \\
 
Many researches are currently using DepMap datasets to identify relatively small sets of genes which are responsible of cancers growth\footnote{Background material: \url{https://depmap.org/portal/publications/}}. This procedure is often driven by medical knowledge,  which we do not possess,  together with some rough measures of importance.  Being Maths student,  we instead base our research on statistical models and on the hypothesis that, if a given classifier is able to distinguish different types of cancer, then the most relevant genes are the most important features for that classifier (the meaning of "important features" will be clarified later).  Of course,  selecting few truly significant genes has outstanding implications in the medical field: building faster diagnosis tools and synthesizing less toxic drugs are only two examples. 


\section{Dataset}
We used two public datasets from the DepMap Public 21Q3 database,  released on August 2021\footnote{Download dataset from DepMap:  \url{https://depmap.org/portal/download/}}:
\begin{itemize}
	\item[D1] \textit{CRISPR\_gene\_dependency.csv},  which contains $1.032$ cancer cell lines characterised by $17.393$ gene scoring results
	\item[D2] \textit{sample\_info.csv}, which contains cell lines information,  such as primary disease and sample collection site
\end{itemize}
Data were collected from real patients and successively processed,  so that element $(i, j)$ of this $(1.032 \times 17.393)$-data frame is the probability that knocking out gene $j$ has a real depletion effect on the $i$-th cell.
Before proceeding with our analysis, we removed missing values, which affected only 10 rows coming from different tumours, and we looked for weird observation. In particular, we found $2$ "Non-Cancerous" and $6$ "Engineered" cells. The first can be reasonably discarded, whereas the latter requires a little care. Engineered cells are synthetically modified samples in lab and, here, they are manly associated to the Eye sample collection site. We decided to keep them and associate them to the cancer corresponding to their site. \\

\begin{wrapfigure}{r}{0.65\textwidth}
	\includegraphics[width=0.65\textwidth]{plot1.png}
	\captionof{figure}{Cancer classes}\label{fig1}
\end{wrapfigure}

We  grouped the various cancer types in 10 classes according to common medical knowledge\footnote{Cancer types grouped by body location: \url{https://www.cancer.gov/types/by-body-location}} and we obtained classes as reported in Figure \ref{fig1}.  "Eye" is the smallest one as there are only $16$ observations, $5$ of which labelled as "Enginereed". On the other hand, "Gastrointestinal" is the largest group and it comprehends $7$ types of cancer,  making this group quite heterogeneous.
We decide to investigate two binary classification problems,  Blood vs Rest and Lung vs Rest, and the multiclass problem. We chose Lung because of the nature of such a class: it is the most numerous group composed only by Lung cancer samples. The choice of Blood was instead driven by some underlying biological knowledge. In fact,  Blood cancer is quite different from other tumours because
\begin{itemize}
	\item blood is in the whole body
	\item Leukemia, Lymphom and Myeloma are the main kinds of cancer but they all affect white blood cells
	\item not all blood cancers require a treatment, just periodical monitoring
\end{itemize} 


\section{Methods}
This section deals with a brief explanation of the algorithms we used.  Each procedure is characterized by three steps:
\begin{enumerate}
\item fit the model using all the features
\item identify the most important ones based on some measure of importance
\item use these genes to fit a reduced version of the classifier and see how it performs
\end{enumerate}

\subsection{Random Forest}
We started by fitting Random Forest (RF) models as they frequently performs well on imbalanced data and correlated high-dimensional data.  Once the model had been fitted,  we used Variable Importance of RFs to identify the most important features. This measure is calculated in three steps.  First, prediction accuracy are measured on the out-of-bag samples. Then the values of the variable are randomly shuffled, keeping all other variables the same. Finally, the decrease in prediction accuracy on the shuffled data is measured and the mean decrease in accuracy across all trees is reported. Intuitively, the random shuffling means that, on average, the shuffled variable has no predictive power. This importance is a measure of by how much removing a variable decreases accuracy, and vice versa. \\
We used this Variable Importance measure to select the most important feature in two different ways:
%\begin{wrapfigure}{l}{0.3\textwidth}
%	\includegraphics[width=0.3\textwidth]{Boruta-algo.jpg}\label{fig2}
%\end{wrapfigure}

\begin{itemize}
	\item \textit{Cross-validation}. We perfomed a 5-fold cross-validation on the model, selected the top most important features of each model and finally averaged.
	\item \textit{Boruta algorithm}\footnote{Boruta: \url{https://www.researchgate.net/publication/220443685_Boruta_-_A_System_for_Feature_Selection}}: Boruta repeatedly measures feature importance then carries out statistical tests to screen out the features which are irrelevant. 
\end{itemize} 


\subsection{SVM-Lasso}
Support vector machines (SVM) are based on the idea of finding a hyperplane that best separate classes. Here, we combine this method with the classical Lasso penalty,  so that the objective function to be minimized is:
\begin{equation*}
	\dfrac{1}{n} \sum_{i=1}^n hingeLoss(y_i(x_i w + t)) + \lambda \sum_{j=1}^p |w_j|  \qquad	\text{where} \qquad  hingeLoss(z) = max\{0, 1-z\}
\end{equation*}
where the parameter $\lambda$ is chosen via cross-validation.  We then obtain the sparsity in predictors: some $w_i$ are shrunken all the way to zero while the ones that are not will be used to identify the important features. 

\subsection{Neural Networks}
Neural Networks (NN) are efficient models to capture non-linear relationship between the predictors and the target variables.  In this context we trained NN with two hidden layers of width 400 and 300 respectively using  the \textit{ReLU} function as activation  for the hidden layer and the \textit{sigmoid} in the binary classification problem and the \textit{softmax} in the multiclass one for the output layer.  Furthermore,  being the binary classification problems a little unbalanced,  a part from the usual \textit{Cross Entropy} loss function we also used the \textit{Focal Loss}, which is defined as $$ FL(z) = \alpha \cdot (1 - z)^{\gamma} \log{z} \text{,  \hspace{3pt} with }z \in [0,1]  \text{ and } \alpha,  \gamma \geq 0$$ in order to focus learning on hard negative examples.  Note that \textit{Focal Loss} can be extended and used in mutliclass classification tasks as well\footnote{Focal Loss: \url{https://arxiv.org/pdf/1708.02002.pdf}}.  Once this NNs were fitted,  we ranked the variable according to the Olden Importance measure\footnote{Olden Importance: \url{https://depts.washington.edu/oldenlab/wordpress/wp-content/uploads/2013/03/EcologicalModelling_2004.pdf}}, selected the first 120 and used them to construct a reduced version of the classifier.



\section{Results}
\subsection{Binary classifications}
Before starting our Binary classifications on Blood and Lung cancer, we ran a Principal Component Analysis to gain some valuable insights. We noticed a surprising result: even if observations formed a cloud of points, Blood cancer cells were mainly concentrated in just one part of the 3-dimensional plot. The same did not happen for Lung cancer observations.

We then moved to RF classifiers. 





\subsection{Multiclass classification}


\section{Conclusion and future works}
\begin{itemize}
\item reiterate key findings
\item limitations: data high correlated,  no medical knowledge,  no meaning at selected genes
\item suggest for the future: work with a med student,  try other classes or methods,  improve lung in some way
\end{itemize}


\end{document}